% Options for packages loaded elsewhere
\PassOptionsToPackage{unicode}{hyperref}
\PassOptionsToPackage{hyphens}{url}
\documentclass[
]{article}
\usepackage{xcolor}
\usepackage[margin=1in]{geometry}
\usepackage{amsmath,amssymb}
\setcounter{secnumdepth}{-\maxdimen} % remove section numbering
\usepackage{iftex}
\ifPDFTeX
  \usepackage[T1]{fontenc}
  \usepackage[utf8]{inputenc}
  \usepackage{textcomp} % provide euro and other symbols
\else % if luatex or xetex
  \usepackage{unicode-math} % this also loads fontspec
  \defaultfontfeatures{Scale=MatchLowercase}
  \defaultfontfeatures[\rmfamily]{Ligatures=TeX,Scale=1}
\fi
\usepackage{lmodern}
\ifPDFTeX\else
  % xetex/luatex font selection
\fi
% Use upquote if available, for straight quotes in verbatim environments
\IfFileExists{upquote.sty}{\usepackage{upquote}}{}
\IfFileExists{microtype.sty}{% use microtype if available
  \usepackage[]{microtype}
  \UseMicrotypeSet[protrusion]{basicmath} % disable protrusion for tt fonts
}{}
\makeatletter
\@ifundefined{KOMAClassName}{% if non-KOMA class
  \IfFileExists{parskip.sty}{%
    \usepackage{parskip}
  }{% else
    \setlength{\parindent}{0pt}
    \setlength{\parskip}{6pt plus 2pt minus 1pt}}
}{% if KOMA class
  \KOMAoptions{parskip=half}}
\makeatother
\usepackage{color}
\usepackage{fancyvrb}
\newcommand{\VerbBar}{|}
\newcommand{\VERB}{\Verb[commandchars=\\\{\}]}
\DefineVerbatimEnvironment{Highlighting}{Verbatim}{commandchars=\\\{\}}
% Add ',fontsize=\small' for more characters per line
\usepackage{framed}
\definecolor{shadecolor}{RGB}{248,248,248}
\newenvironment{Shaded}{\begin{snugshade}}{\end{snugshade}}
\newcommand{\AlertTok}[1]{\textcolor[rgb]{0.94,0.16,0.16}{#1}}
\newcommand{\AnnotationTok}[1]{\textcolor[rgb]{0.56,0.35,0.01}{\textbf{\textit{#1}}}}
\newcommand{\AttributeTok}[1]{\textcolor[rgb]{0.13,0.29,0.53}{#1}}
\newcommand{\BaseNTok}[1]{\textcolor[rgb]{0.00,0.00,0.81}{#1}}
\newcommand{\BuiltInTok}[1]{#1}
\newcommand{\CharTok}[1]{\textcolor[rgb]{0.31,0.60,0.02}{#1}}
\newcommand{\CommentTok}[1]{\textcolor[rgb]{0.56,0.35,0.01}{\textit{#1}}}
\newcommand{\CommentVarTok}[1]{\textcolor[rgb]{0.56,0.35,0.01}{\textbf{\textit{#1}}}}
\newcommand{\ConstantTok}[1]{\textcolor[rgb]{0.56,0.35,0.01}{#1}}
\newcommand{\ControlFlowTok}[1]{\textcolor[rgb]{0.13,0.29,0.53}{\textbf{#1}}}
\newcommand{\DataTypeTok}[1]{\textcolor[rgb]{0.13,0.29,0.53}{#1}}
\newcommand{\DecValTok}[1]{\textcolor[rgb]{0.00,0.00,0.81}{#1}}
\newcommand{\DocumentationTok}[1]{\textcolor[rgb]{0.56,0.35,0.01}{\textbf{\textit{#1}}}}
\newcommand{\ErrorTok}[1]{\textcolor[rgb]{0.64,0.00,0.00}{\textbf{#1}}}
\newcommand{\ExtensionTok}[1]{#1}
\newcommand{\FloatTok}[1]{\textcolor[rgb]{0.00,0.00,0.81}{#1}}
\newcommand{\FunctionTok}[1]{\textcolor[rgb]{0.13,0.29,0.53}{\textbf{#1}}}
\newcommand{\ImportTok}[1]{#1}
\newcommand{\InformationTok}[1]{\textcolor[rgb]{0.56,0.35,0.01}{\textbf{\textit{#1}}}}
\newcommand{\KeywordTok}[1]{\textcolor[rgb]{0.13,0.29,0.53}{\textbf{#1}}}
\newcommand{\NormalTok}[1]{#1}
\newcommand{\OperatorTok}[1]{\textcolor[rgb]{0.81,0.36,0.00}{\textbf{#1}}}
\newcommand{\OtherTok}[1]{\textcolor[rgb]{0.56,0.35,0.01}{#1}}
\newcommand{\PreprocessorTok}[1]{\textcolor[rgb]{0.56,0.35,0.01}{\textit{#1}}}
\newcommand{\RegionMarkerTok}[1]{#1}
\newcommand{\SpecialCharTok}[1]{\textcolor[rgb]{0.81,0.36,0.00}{\textbf{#1}}}
\newcommand{\SpecialStringTok}[1]{\textcolor[rgb]{0.31,0.60,0.02}{#1}}
\newcommand{\StringTok}[1]{\textcolor[rgb]{0.31,0.60,0.02}{#1}}
\newcommand{\VariableTok}[1]{\textcolor[rgb]{0.00,0.00,0.00}{#1}}
\newcommand{\VerbatimStringTok}[1]{\textcolor[rgb]{0.31,0.60,0.02}{#1}}
\newcommand{\WarningTok}[1]{\textcolor[rgb]{0.56,0.35,0.01}{\textbf{\textit{#1}}}}
\usepackage{graphicx}
\makeatletter
\newsavebox\pandoc@box
\newcommand*\pandocbounded[1]{% scales image to fit in text height/width
  \sbox\pandoc@box{#1}%
  \Gscale@div\@tempa{\textheight}{\dimexpr\ht\pandoc@box+\dp\pandoc@box\relax}%
  \Gscale@div\@tempb{\linewidth}{\wd\pandoc@box}%
  \ifdim\@tempb\p@<\@tempa\p@\let\@tempa\@tempb\fi% select the smaller of both
  \ifdim\@tempa\p@<\p@\scalebox{\@tempa}{\usebox\pandoc@box}%
  \else\usebox{\pandoc@box}%
  \fi%
}
% Set default figure placement to htbp
\def\fps@figure{htbp}
\makeatother
\setlength{\emergencystretch}{3em} % prevent overfull lines
\providecommand{\tightlist}{%
  \setlength{\itemsep}{0pt}\setlength{\parskip}{0pt}}
\usepackage{bookmark}
\IfFileExists{xurl.sty}{\usepackage{xurl}}{} % add URL line breaks if available
\urlstyle{same}
\hypersetup{
  pdftitle={Monte Carlo Methods: Sampling Algorithms},
  pdfauthor={Calum Smith},
  hidelinks,
  pdfcreator={LaTeX via pandoc}}

\title{Monte Carlo Methods: Sampling Algorithms}
\author{Calum Smith}
\date{2026-01-25}

\begin{document}
\maketitle

\begin{Shaded}
\begin{Highlighting}[]
\FunctionTok{rm}\NormalTok{(}\AttributeTok{list=}\FunctionTok{ls}\NormalTok{())}
\FunctionTok{library}\NormalTok{(ggplot2)}
\FunctionTok{library}\NormalTok{(dplyr)}

\FunctionTok{set.seed}\NormalTok{(}\DecValTok{20}\NormalTok{)}
\end{Highlighting}
\end{Shaded}

\section{1. Inverse Transform
Sampling}\label{inverse-transform-sampling}

\subsection{Theoretical Derivation}\label{theoretical-derivation}

We aim to sample from the random variable \(Y = X^\alpha\), where \(X\)
follows a Laplace distribution. To do this, we first derive the
Cumulative Distribution Function (CDF) of \(Y\). Given \(F_X(x)\) for a
standard Laplace distribution, the CDF of \(Y\) is:

\[
F_Y(y) = \mathbb{P}(Y \le y) = F_X(y^{1/\alpha}) = 
\begin{cases} 
\frac{1}{2}e^{y^{1/\alpha}} & \text{if } y < 0 \\
1 - \frac{1}{2}e^{-y^{1/\alpha}} & \text{if } y \ge 0 
\end{cases}
\]

Using the Probability Integral Transform, we invert this CDF to find the
sampling function:

\[
F_Y^{-1}(u) = 
\begin{cases} 
[\ln(2u)]^\alpha & \text{if } 0 < u \le 0.5 \\
[-\ln(2(1-u))]^\alpha & \text{if } 0.5 < u < 1 
\end{cases}
\]

\subsection{Implementation}\label{implementation}

The following code implements this inverse CDF method for various values
of \(\alpha\).

\begin{Shaded}
\begin{Highlighting}[]
\CommentTok{\# Define the Sampling Function}
\NormalTok{gicdf.y }\OtherTok{\textless{}{-}} \ControlFlowTok{function}\NormalTok{(u, alpha) \{}
\NormalTok{  x }\OtherTok{\textless{}{-}} \FunctionTok{numeric}\NormalTok{(}\FunctionTok{length}\NormalTok{(u))}
\NormalTok{  ind.neg }\OtherTok{\textless{}{-}}\NormalTok{ (u }\SpecialCharTok{\textless{}=} \FloatTok{0.5}\NormalTok{)}
\NormalTok{  x[ind.neg] }\OtherTok{\textless{}{-}} \FunctionTok{log}\NormalTok{(}\DecValTok{2} \SpecialCharTok{*}\NormalTok{ u[ind.neg])}
\NormalTok{  x[}\SpecialCharTok{!}\NormalTok{ind.neg] }\OtherTok{\textless{}{-}} \SpecialCharTok{{-}}\FunctionTok{log}\NormalTok{(}\DecValTok{2} \SpecialCharTok{*}\NormalTok{ (}\DecValTok{1} \SpecialCharTok{{-}}\NormalTok{ u[}\SpecialCharTok{!}\NormalTok{ind.neg]))}
\NormalTok{  y }\OtherTok{\textless{}{-}} \FunctionTok{sign}\NormalTok{(x) }\SpecialCharTok{*}\NormalTok{ (}\FunctionTok{abs}\NormalTok{(x))}\SpecialCharTok{\^{}}\NormalTok{alpha}
  \FunctionTok{return}\NormalTok{(y)}
\NormalTok{\}}

\CommentTok{\# Generate the Sample Data}
\NormalTok{n.samples }\OtherTok{\textless{}{-}} \DecValTok{1000}
\NormalTok{alphas }\OtherTok{\textless{}{-}} \FunctionTok{c}\NormalTok{(}\DecValTok{1}\NormalTok{, }\DecValTok{3}\NormalTok{, }\DecValTok{5}\NormalTok{)}

\NormalTok{samples }\OtherTok{\textless{}{-}} \FunctionTok{lapply}\NormalTok{(alphas, }\ControlFlowTok{function}\NormalTok{(a) \{}
\NormalTok{  u.values }\OtherTok{\textless{}{-}} \FunctionTok{runif}\NormalTok{(n.samples)}
\NormalTok{  alpha.vec }\OtherTok{\textless{}{-}} \FunctionTok{rep}\NormalTok{(a, n.samples)}
  \FunctionTok{gicdf.y}\NormalTok{(u.values, alpha.vec)}
\NormalTok{\})}
\end{Highlighting}
\end{Shaded}

\subsubsection{Plot of the target
distributions}\label{plot-of-the-target-distributions}

To verify the distribution, we can plot a histogram of the generated
samples and overlay the theoretical probability density function (PDF)
derived in part (a).

For \(\alpha = 1\), the density of \(Y\) is the Laplace distribution as
described above so we get the characteristic tent shape centred at 0. As
we increase the value of \(\alpha\), the density of \(Y\) gets stretched
and the tails become very heavy. This makes sense because for example,
if \(X=3\) then under the transformation with \(\alpha=3\) we have that
\(Y=3^3=27\) and so we are spreading out the density for values of
\(X\). Moreover, the values close to 0 get even closer due to \(\alpha\)
which causes the spike as \(\alpha\) increases.

\begin{Shaded}
\begin{Highlighting}[]
\CommentTok{\# Define the theoretical PDF of Y}
\NormalTok{d\_Y }\OtherTok{\textless{}{-}} \ControlFlowTok{function}\NormalTok{(y, alpha) \{}
\NormalTok{  term1 }\OtherTok{\textless{}{-}} \DecValTok{1} \SpecialCharTok{/}\NormalTok{ (}\DecValTok{2} \SpecialCharTok{*}\NormalTok{ alpha)}
\NormalTok{  term2 }\OtherTok{\textless{}{-}} \FunctionTok{abs}\NormalTok{(y)}\SpecialCharTok{\^{}}\NormalTok{((}\DecValTok{1} \SpecialCharTok{{-}}\NormalTok{ alpha) }\SpecialCharTok{/}\NormalTok{ alpha)}
\NormalTok{  term3 }\OtherTok{\textless{}{-}} \FunctionTok{exp}\NormalTok{(}\SpecialCharTok{{-}}\FunctionTok{abs}\NormalTok{(y)}\SpecialCharTok{\^{}}\NormalTok{(}\DecValTok{1} \SpecialCharTok{/}\NormalTok{ alpha))}
  
  
\NormalTok{  density }\OtherTok{\textless{}{-}}\NormalTok{ term1 }\SpecialCharTok{*}\NormalTok{ term2 }\SpecialCharTok{*}\NormalTok{ term3}
\NormalTok{  density[y }\SpecialCharTok{==} \DecValTok{0} \SpecialCharTok{\&}\NormalTok{ alpha }\SpecialCharTok{\textgreater{}} \DecValTok{1}\NormalTok{] }\OtherTok{\textless{}{-}} \DecValTok{0}
\NormalTok{  density[y }\SpecialCharTok{==} \DecValTok{0} \SpecialCharTok{\&}\NormalTok{ alpha }\SpecialCharTok{==} \DecValTok{1}\NormalTok{] }\OtherTok{\textless{}{-}} \FloatTok{0.5}
  
  \FunctionTok{return}\NormalTok{(density)}
\NormalTok{\}}

\NormalTok{plot\_data }\OtherTok{\textless{}{-}} \FunctionTok{mapply}\NormalTok{(}\ControlFlowTok{function}\NormalTok{(s, a) }\FunctionTok{data.frame}\NormalTok{(}\AttributeTok{y =}\NormalTok{ s, }\AttributeTok{alpha =}\NormalTok{ a),}
\NormalTok{                    samples, alphas, }\AttributeTok{SIMPLIFY =} \ConstantTok{FALSE}\NormalTok{) }\SpecialCharTok{\%\textgreater{}\%}
  \FunctionTok{bind\_rows}\NormalTok{() }\SpecialCharTok{\%\textgreater{}\%}
  \FunctionTok{mutate}\NormalTok{(}\AttributeTok{alpha\_label =} \FunctionTok{factor}\NormalTok{(}\FunctionTok{paste}\NormalTok{(}\StringTok{"alpha ="}\NormalTok{, alpha)))}

\CommentTok{\# Trim the data for better visualisation}
\NormalTok{trimmed\_plot\_data }\OtherTok{\textless{}{-}}\NormalTok{ plot\_data }\SpecialCharTok{\%\textgreater{}\%}
  \FunctionTok{group\_by}\NormalTok{(alpha\_label) }\SpecialCharTok{\%\textgreater{}\%}
  \FunctionTok{mutate}\NormalTok{(}\AttributeTok{q\_low =} \FunctionTok{quantile}\NormalTok{(y, }\FloatTok{0.005}\NormalTok{),}
         \AttributeTok{q\_high =} \FunctionTok{quantile}\NormalTok{(y, }\FloatTok{0.995}\NormalTok{)) }\SpecialCharTok{\%\textgreater{}\%}
  \FunctionTok{filter}\NormalTok{(y }\SpecialCharTok{\textgreater{}=}\NormalTok{ q\_low }\SpecialCharTok{\&}\NormalTok{ y }\SpecialCharTok{\textless{}=}\NormalTok{ q\_high) }\SpecialCharTok{\%\textgreater{}\%}
  \FunctionTok{ungroup}\NormalTok{()}

\CommentTok{\# Create the theoretical density data for the lines}
\NormalTok{theory\_data }\OtherTok{\textless{}{-}}\NormalTok{ trimmed\_plot\_data }\SpecialCharTok{\%\textgreater{}\%}
  \FunctionTok{group\_by}\NormalTok{(alpha\_label, alpha) }\SpecialCharTok{\%\textgreater{}\%}
  \FunctionTok{summarise}\NormalTok{(}\AttributeTok{min\_y =} \FunctionTok{min}\NormalTok{(y), }\AttributeTok{max\_y =} \FunctionTok{max}\NormalTok{(y)) }\SpecialCharTok{\%\textgreater{}\%}
  \FunctionTok{group\_by}\NormalTok{(alpha\_label, alpha) }\SpecialCharTok{\%\textgreater{}\%}
  \FunctionTok{summarise}\NormalTok{(}\AttributeTok{y =} \FunctionTok{seq}\NormalTok{(min\_y, max\_y, }\AttributeTok{length.out =} \DecValTok{501}\NormalTok{)) }\SpecialCharTok{\%\textgreater{}\%}
  \FunctionTok{mutate}\NormalTok{(}\AttributeTok{pdf =} \FunctionTok{d\_Y}\NormalTok{(y, alpha)) }\SpecialCharTok{\%\textgreater{}\%}
  \FunctionTok{ungroup}\NormalTok{()}

\CommentTok{\# Generate the plot}
\FunctionTok{ggplot}\NormalTok{(trimmed\_plot\_data, }\FunctionTok{aes}\NormalTok{(}\AttributeTok{x =}\NormalTok{ y)) }\SpecialCharTok{+}
  
  \CommentTok{\# Histogram of the samples}
  \FunctionTok{geom\_histogram}\NormalTok{(}\FunctionTok{aes}\NormalTok{(}\AttributeTok{y =} \FunctionTok{after\_stat}\NormalTok{(density)), }
                 \AttributeTok{bins =} \DecValTok{100}\NormalTok{, }
                 \AttributeTok{fill =} \StringTok{"\#0072B2"}\NormalTok{, }
                 \AttributeTok{alpha =} \FloatTok{0.6}\NormalTok{,}
                 \AttributeTok{color =} \StringTok{"white"}\NormalTok{,}
                 \AttributeTok{linewidth =} \FloatTok{0.2}\NormalTok{) }\SpecialCharTok{+}
  
  \CommentTok{\# Theoretical density line}
  \FunctionTok{geom\_line}\NormalTok{(}\AttributeTok{data =}\NormalTok{ theory\_data, }
            \FunctionTok{aes}\NormalTok{(}\AttributeTok{x =}\NormalTok{ y, }\AttributeTok{y =}\NormalTok{ pdf), }
            \AttributeTok{color =} \StringTok{"\#D55E00"}\NormalTok{, }
            \AttributeTok{linewidth =} \FloatTok{1.2}\NormalTok{) }\SpecialCharTok{+}
  
  \FunctionTok{facet\_wrap}\NormalTok{(}\SpecialCharTok{\textasciitilde{}}\NormalTok{ alpha\_label, }\AttributeTok{scales =} \StringTok{"free"}\NormalTok{) }\SpecialCharTok{+}
  
  \FunctionTok{labs}\NormalTok{(}
    \AttributeTok{title =} \StringTok{"Sample Histograms vs. Theoretical Density of Y"}\NormalTok{,}
    \AttributeTok{subtitle =} \StringTok{"Samples (blue) and Theoretical PDF (orange)"}\NormalTok{,}
    \AttributeTok{x =} \StringTok{"Y"}\NormalTok{,}
    \AttributeTok{y =} \StringTok{"Density"}
\NormalTok{  ) }\SpecialCharTok{+}
  \FunctionTok{theme\_bw}\NormalTok{(}\AttributeTok{base\_size =} \DecValTok{14}\NormalTok{) }\SpecialCharTok{+}
  \FunctionTok{theme}\NormalTok{(}
    \AttributeTok{strip.background =} \FunctionTok{element\_rect}\NormalTok{(}\AttributeTok{fill =} \StringTok{"grey90"}\NormalTok{),}
    \AttributeTok{strip.text =} \FunctionTok{element\_text}\NormalTok{(}\AttributeTok{face =} \StringTok{"bold"}\NormalTok{)}
\NormalTok{  )}
\end{Highlighting}
\end{Shaded}

\pandocbounded{\includegraphics[keepaspectratio]{Monte_Carlo_Methods_Sampling_Algorithms_files/figure-latex/unnamed-chunk-3-1.pdf}}

\section{2. Optimization of Rejection
Sampling}\label{optimization-of-rejection-sampling}

For a distribution with PDF \(q(x)\) to be a valid proposal distribution
for the target distribution with PDF \(p(x)\), there must exist a finite
constant \(M\) such that the ratio \(p(x)/q(x)\) is bounded for all
\(x\): \begin{align*}
            \frac{p(x)}{q(x)} \le M \quad \text{for all } x \in (-\infty,\infty)
        \end{align*} Where our target is
\(p(x) = \frac{1}{2\alpha}x^{\frac{1}{\alpha}-1} e^{-{|x^{\frac{1}{\alpha}}|}}\)
and the proposal \(q(x)\) is the \(t\)-distribution given in the
question. In order to analyse the behaviour of the densities we need to
write them both with respect to \(x\). To determine the value of
\(\alpha\) we need to examine the behaviour of ratio of the densities at

\begin{enumerate}[label=-]
            \item $x \to 0$ since this is the value at which we get the mode for both densities
            \item $x \to \infty$ to check how the tails behave
        \end{enumerate}

Checking each limit in turn we get: \begin{align*}
            \lim_{x \to 0} \frac{p(x)}{q(x)} = \lim_{x \to 0} \frac{\frac{1}{2\alpha}x^{\frac{1}{\alpha}-1} e^{-{|x^{\frac{1}{\alpha}}|}}}{q(x)}
        \end{align*} We know that the \(t\)-distribution with \(\nu\)
degrees of freedom has a finite constant mode for any
\(\nu \in (0, \infty)\) so we can write \(\lim_{x\to 0}q(x)=C\) for some
\(C>0\). Now, we have that the limit reduces to \begin{align*}
            \lim_{x \to 0} \frac{p(x)}{q(x)} = \frac{1}{2\alpha C} \lim_{x\to 0} x^{\frac{1}{\alpha} - 1}
        \end{align*} So the limit of the ratio depends on the sign of
\(\frac{1}{\alpha} - 1\). Now we consider two cases:

\begin{enumerate}
            \item $\alpha=1$: In this case the limit reduces to $1/2C$ which is a finite positive constant. For the case when $x \to \infty$, the ratio goes to 0 because the exponential tail of $p(x)$ decays faster that the polynomial tail of $q(x)$. Since the ratio is a continuous function that is finite at $x=0$ and tends to 0 at infinity, it is bounded over its entire domain. Hence, a $t$-distribution is a valid proposal for $\alpha=1$.
            \item $\alpha > 1$: In this case, $\frac{1}{\alpha} - 1 < 0$. This means that
            \begin{align*}
                \frac{1}{2\alpha C} \lim_{x\to 0} x^{\frac{1}{\alpha} - 1} = \infty
            \end{align*}
            so the ratio is unbounded. This means we cannot find a finite $M$ to satisfy the condition above for all $x$ close to 0. Hence, a $t$-distribution would not be a valid proposal for $\alpha > 1$.
        \end{enumerate}

Hence, for rejection sampling to work we need to choose \(\alpha=1\).

To maximise the efficiency of the rejection sampler we need to minimise
the constant \(M\) where \(M = \sup_x \frac{f(x)}{g(x)}\). Since \(M\)
will depend on the parameter \(\nu\) of our proposal \(g(x)\), we are
looking for the value \(\nu^*\) that minimises \(M(\nu)\). To find the
maximum of this ratio we differentiate with respect to \(x\) and set to
0. Once we do this, we find that the critical values are at \(x=1\) and
\(x=\nu\). Hence, \begin{align*}
            M(\nu) = \frac{\sqrt{\nu \pi}\Gamma(\frac{\nu}{2})}{2\Gamma(\frac{\nu + 1}{2})} \max(1, e^{-\nu}(1+\nu)^{(\nu +1)/2}
        \end{align*} Then to find \(\nu\) we need to numerically
calculate it using R. After doing this we find that \(\nu \approx 3.92\)
minimises \(M(\nu)\).

\begin{Shaded}
\begin{Highlighting}[]
\CommentTok{\# Find optimal nu}
\NormalTok{get\_M }\OtherTok{\textless{}{-}} \ControlFlowTok{function}\NormalTok{(nu) \{}
\NormalTok{  log\_C\_nu }\OtherTok{\textless{}{-}}\NormalTok{ (}\FloatTok{0.5} \SpecialCharTok{*} \FunctionTok{log}\NormalTok{(nu }\SpecialCharTok{*}\NormalTok{ pi) }\SpecialCharTok{+} 
                 \FunctionTok{lgamma}\NormalTok{(nu }\SpecialCharTok{/} \DecValTok{2}\NormalTok{) }\SpecialCharTok{{-}} 
                 \FunctionTok{log}\NormalTok{(}\DecValTok{2}\NormalTok{) }\SpecialCharTok{{-}} 
                 \FunctionTok{lgamma}\NormalTok{((nu }\SpecialCharTok{+} \DecValTok{1}\NormalTok{) }\SpecialCharTok{/} \DecValTok{2}\NormalTok{))}
\NormalTok{  C\_nu }\OtherTok{\textless{}{-}} \FunctionTok{exp}\NormalTok{(log\_C\_nu)}
\NormalTok{  h\_at\_0 }\OtherTok{\textless{}{-}} \FloatTok{1.0}
\NormalTok{  log\_h\_at\_1 }\OtherTok{\textless{}{-}} \SpecialCharTok{{-}}\FloatTok{1.0} \SpecialCharTok{+}\NormalTok{ ((nu }\SpecialCharTok{+} \DecValTok{1}\NormalTok{) }\SpecialCharTok{/} \DecValTok{2}\NormalTok{) }\SpecialCharTok{*} \FunctionTok{log}\NormalTok{(}\DecValTok{1} \SpecialCharTok{+} \DecValTok{1}\SpecialCharTok{/}\NormalTok{nu)}
\NormalTok{  h\_at\_1 }\OtherTok{\textless{}{-}} \FunctionTok{exp}\NormalTok{(log\_h\_at\_1)}
\NormalTok{  log\_h\_at\_nu }\OtherTok{\textless{}{-}} \SpecialCharTok{{-}}\NormalTok{nu }\SpecialCharTok{+}\NormalTok{ ((nu }\SpecialCharTok{+} \DecValTok{1}\NormalTok{) }\SpecialCharTok{/} \DecValTok{2}\NormalTok{) }\SpecialCharTok{*} \FunctionTok{log}\NormalTok{(}\DecValTok{1} \SpecialCharTok{+}\NormalTok{ nu)}
\NormalTok{  h\_at\_nu }\OtherTok{\textless{}{-}} \FunctionTok{exp}\NormalTok{(log\_h\_at\_nu)}
\NormalTok{  sup\_h }\OtherTok{\textless{}{-}} \FunctionTok{max}\NormalTok{(h\_at\_0, h\_at\_1, h\_at\_nu)}
  
  \FunctionTok{return}\NormalTok{(C\_nu }\SpecialCharTok{*}\NormalTok{ sup\_h)}
\NormalTok{\}}

\CommentTok{\# Find the optimal nu}
\NormalTok{opt\_result }\OtherTok{\textless{}{-}} \FunctionTok{optimize}\NormalTok{(get\_M, }\AttributeTok{interval =} \FunctionTok{c}\NormalTok{(}\DecValTok{1}\NormalTok{, }\DecValTok{10}\NormalTok{))}
\NormalTok{nu\_optimal }\OtherTok{\textless{}{-}}\NormalTok{ opt\_result}\SpecialCharTok{$}\NormalTok{minimum}
\NormalTok{M\_optimal }\OtherTok{\textless{}{-}}\NormalTok{ opt\_result}\SpecialCharTok{$}\NormalTok{objective}

\CommentTok{\# Generate data for plotting}
\NormalTok{nu\_values }\OtherTok{\textless{}{-}} \FunctionTok{seq}\NormalTok{(}\DecValTok{1}\NormalTok{, }\DecValTok{10}\NormalTok{, }\AttributeTok{by =} \FloatTok{0.05}\NormalTok{)}
\NormalTok{M\_values }\OtherTok{\textless{}{-}} \FunctionTok{sapply}\NormalTok{(nu\_values, get\_M)}

\CommentTok{\# Plot}
\FunctionTok{plot}\NormalTok{(nu\_values, M\_values, }
     \AttributeTok{type =} \StringTok{"l"}\NormalTok{, }
     \AttributeTok{col =} \StringTok{"blue"}\NormalTok{, }
     \AttributeTok{lwd =} \DecValTok{2}\NormalTok{,}
     \AttributeTok{main =} \StringTok{"Effect of nu on the Rejection Sampling Constant M"}\NormalTok{,}
     \AttributeTok{xlab =} \StringTok{"Degrees of Freedom"}\NormalTok{,}
     \AttributeTok{ylab =} \StringTok{"Sampling Constant M"}\NormalTok{,}
     \AttributeTok{panel.first =} \FunctionTok{grid}\NormalTok{()}
\NormalTok{)}

\FunctionTok{points}\NormalTok{(nu\_optimal, M\_optimal, }\AttributeTok{col =} \StringTok{"red"}\NormalTok{, }\AttributeTok{pch =} \DecValTok{19}\NormalTok{, }\AttributeTok{cex =} \FloatTok{1.5}\NormalTok{)}

\FunctionTok{text}\NormalTok{(nu\_optimal, M\_optimal, }
     \AttributeTok{label =} \FunctionTok{sprintf}\NormalTok{(}\StringTok{"Optimal nu = \%.2f}\SpecialCharTok{\textbackslash{}n}\StringTok{M = \%.4f"}\NormalTok{, nu\_optimal, M\_optimal),}
     \AttributeTok{pos =} \DecValTok{3}\NormalTok{,}
     \AttributeTok{col =} \StringTok{"red"}\NormalTok{,}
     \AttributeTok{cex =} \FloatTok{0.9}\NormalTok{)}
\end{Highlighting}
\end{Shaded}

\pandocbounded{\includegraphics[keepaspectratio]{Monte_Carlo_Methods_Sampling_Algorithms_files/figure-latex/unnamed-chunk-4-1.pdf}}

\subsubsection{Implemeting the rejection sampler for the optimal
nu}\label{implemeting-the-rejection-sampler-for-the-optimal-nu}

\begin{Shaded}
\begin{Highlighting}[]
\CommentTok{\# Define varphi}
\NormalTok{phi\_laplace }\OtherTok{\textless{}{-}} \ControlFlowTok{function}\NormalTok{(x) \{}
  \FunctionTok{return}\NormalTok{(}\FloatTok{0.5} \SpecialCharTok{*} \FunctionTok{exp}\NormalTok{(}\SpecialCharTok{{-}}\FunctionTok{abs}\NormalTok{(x)))}
\NormalTok{\}}

\CommentTok{\# Implement the Rejection Sampler}

\NormalTok{run\_rejection\_sampler }\OtherTok{\textless{}{-}} \ControlFlowTok{function}\NormalTok{(n\_samples, nu) \{}
\NormalTok{  M }\OtherTok{\textless{}{-}} \FunctionTok{get\_M}\NormalTok{(nu)}
\NormalTok{  samples }\OtherTok{\textless{}{-}} \FunctionTok{numeric}\NormalTok{(n\_samples)}
\NormalTok{  n\_accepted }\OtherTok{\textless{}{-}} \DecValTok{0}
\NormalTok{  n\_proposed }\OtherTok{\textless{}{-}} \DecValTok{0}
  \ControlFlowTok{while}\NormalTok{ (n\_accepted }\SpecialCharTok{\textless{}}\NormalTok{ n\_samples) \{}
\NormalTok{    n\_proposed }\OtherTok{\textless{}{-}}\NormalTok{ n\_proposed }\SpecialCharTok{+} \DecValTok{1}
\NormalTok{    z }\OtherTok{\textless{}{-}} \FunctionTok{rt}\NormalTok{(}\DecValTok{1}\NormalTok{, }\AttributeTok{df =}\NormalTok{ nu)}
\NormalTok{    u }\OtherTok{\textless{}{-}} \FunctionTok{runif}\NormalTok{(}\DecValTok{1}\NormalTok{)}
\NormalTok{    g\_z }\OtherTok{\textless{}{-}} \FunctionTok{dt}\NormalTok{(z, }\AttributeTok{df =}\NormalTok{ nu)}
    \ControlFlowTok{if}\NormalTok{ (g\_z }\SpecialCharTok{\textgreater{}} \DecValTok{0}\NormalTok{) \{}
\NormalTok{      acceptance\_ratio }\OtherTok{\textless{}{-}} \FunctionTok{phi\_laplace}\NormalTok{(z) }\SpecialCharTok{/}\NormalTok{ (M }\SpecialCharTok{*}\NormalTok{ g\_z)}
      \ControlFlowTok{if}\NormalTok{ (u }\SpecialCharTok{\textless{}=}\NormalTok{ acceptance\_ratio) \{}
\NormalTok{        n\_accepted }\OtherTok{\textless{}{-}}\NormalTok{ n\_accepted }\SpecialCharTok{+} \DecValTok{1}
\NormalTok{        samples[n\_accepted] }\OtherTok{\textless{}{-}}\NormalTok{ z}
\NormalTok{      \}}
\NormalTok{    \}}
\NormalTok{  \}}
  
\NormalTok{  experimental\_rate }\OtherTok{\textless{}{-}}\NormalTok{ n\_samples }\SpecialCharTok{/}\NormalTok{ n\_proposed}
  \FunctionTok{return}\NormalTok{(}\FunctionTok{list}\NormalTok{(}
    \AttributeTok{samples =}\NormalTok{ samples,}
    \AttributeTok{experimental\_rate =}\NormalTok{ experimental\_rate,}
    \AttributeTok{theoretical\_rate =} \FloatTok{1.0} \SpecialCharTok{/}\NormalTok{ M,}
    \AttributeTok{M =}\NormalTok{ M,}
    \AttributeTok{n\_proposed =}\NormalTok{ n\_proposed,}
    \AttributeTok{nu\_used =}\NormalTok{ nu}
\NormalTok{  ))}
\NormalTok{\}}

\NormalTok{N\_SAMPLES }\OtherTok{\textless{}{-}} \DecValTok{10000}
\NormalTok{nu\_optimal }\OtherTok{\textless{}{-}} \FloatTok{3.92}

\CommentTok{\# Run the sampler}
\NormalTok{res }\OtherTok{\textless{}{-}} \FunctionTok{run\_rejection\_sampler}\NormalTok{(N\_SAMPLES, nu\_optimal)}

\CommentTok{\# Compare the theoretical and experimental rates}
\FunctionTok{cat}\NormalTok{(}\FunctionTok{sprintf}\NormalTok{(}\StringTok{"  Target samples: \%d}\SpecialCharTok{\textbackslash{}n}\StringTok{"}\NormalTok{, N\_SAMPLES))}
\end{Highlighting}
\end{Shaded}

\begin{verbatim}
##   Target samples: 10000
\end{verbatim}

\begin{Shaded}
\begin{Highlighting}[]
\FunctionTok{cat}\NormalTok{(}\FunctionTok{sprintf}\NormalTok{(}\StringTok{"  Total proposals needed: \%d}\SpecialCharTok{\textbackslash{}n}\StringTok{"}\NormalTok{, res}\SpecialCharTok{$}\NormalTok{n\_proposed))}
\end{Highlighting}
\end{Shaded}

\begin{verbatim}
##   Total proposals needed: 13378
\end{verbatim}

\begin{Shaded}
\begin{Highlighting}[]
\FunctionTok{cat}\NormalTok{(}\FunctionTok{sprintf}\NormalTok{(}\StringTok{"  Theoretical M: \%.4f}\SpecialCharTok{\textbackslash{}n}\StringTok{"}\NormalTok{, res}\SpecialCharTok{$}\NormalTok{M))}
\end{Highlighting}
\end{Shaded}

\begin{verbatim}
##   Theoretical M: 1.3350
\end{verbatim}

\begin{Shaded}
\begin{Highlighting}[]
\FunctionTok{cat}\NormalTok{(}\FunctionTok{sprintf}\NormalTok{(}\StringTok{"  Theoretical Rate (1/M): \%.4f}\SpecialCharTok{\textbackslash{}n}\StringTok{"}\NormalTok{, res}\SpecialCharTok{$}\NormalTok{theoretical\_rate))}
\end{Highlighting}
\end{Shaded}

\begin{verbatim}
##   Theoretical Rate (1/M): 0.7491
\end{verbatim}

\begin{Shaded}
\begin{Highlighting}[]
\FunctionTok{cat}\NormalTok{(}\FunctionTok{sprintf}\NormalTok{(}\StringTok{"  Experimental Rate: \%.4f}\SpecialCharTok{\textbackslash{}n}\StringTok{"}\NormalTok{, res}\SpecialCharTok{$}\NormalTok{experimental\_rate))}
\end{Highlighting}
\end{Shaded}

\begin{verbatim}
##   Experimental Rate: 0.7475
\end{verbatim}

\subsubsection{Question 2 Part (c)}\label{question-2-part-c}

\begin{Shaded}
\begin{Highlighting}[]
\CommentTok{\# Implementing an Importance Sampler}

\CommentTok{\# Define phi}
\NormalTok{phi }\OtherTok{\textless{}{-}} \ControlFlowTok{function}\NormalTok{(t) \{}
  \FunctionTok{exp}\NormalTok{(}\SpecialCharTok{{-}}\NormalTok{t}\SpecialCharTok{\^{}}\DecValTok{4}\NormalTok{)}
\NormalTok{\}}

\CommentTok{\# Importance sampling function}
\NormalTok{is.estimator }\OtherTok{\textless{}{-}} \ControlFlowTok{function}\NormalTok{(}\AttributeTok{n=}\DecValTok{1000}\NormalTok{) \{}
\NormalTok{  t.g }\OtherTok{\textless{}{-}} \FunctionTok{rnorm}\NormalTok{(n)}
\NormalTok{  W }\OtherTok{\textless{}{-}} \FunctionTok{dt}\NormalTok{(t.g, }\AttributeTok{df=}\DecValTok{2}\NormalTok{) }\SpecialCharTok{/} \FunctionTok{dnorm}\NormalTok{(t.g)}
  \FunctionTok{return}\NormalTok{(}\FunctionTok{mean}\NormalTok{(}\FunctionTok{phi}\NormalTok{(t.g)}\SpecialCharTok{*}\NormalTok{ W))}
\NormalTok{\}}

\CommentTok{\# Simple Monte Carlo sampler}
\NormalTok{smc.estimator }\OtherTok{\textless{}{-}} \ControlFlowTok{function}\NormalTok{(}\AttributeTok{n=}\DecValTok{1000}\NormalTok{) \{}
\NormalTok{  t.f }\OtherTok{\textless{}{-}} \FunctionTok{rt}\NormalTok{(n, }\AttributeTok{df=}\DecValTok{2}\NormalTok{)}
  \FunctionTok{return}\NormalTok{(}\FunctionTok{mean}\NormalTok{(}\FunctionTok{phi}\NormalTok{(t.f)))}
\NormalTok{\}}

\CommentTok{\# Run the IS estimator multiple times}
\NormalTok{is.replicate }\OtherTok{\textless{}{-}} \ControlFlowTok{function}\NormalTok{(}\AttributeTok{reps=}\DecValTok{5000}\NormalTok{, }\AttributeTok{n=}\DecValTok{1000}\NormalTok{)\{}
\NormalTok{  estimates }\OtherTok{\textless{}{-}} \FunctionTok{sapply}\NormalTok{(}\DecValTok{1}\SpecialCharTok{:}\NormalTok{reps, }\ControlFlowTok{function}\NormalTok{(i) }\FunctionTok{is.estimator}\NormalTok{(n))}
  \FunctionTok{return}\NormalTok{(estimates)}
\NormalTok{\}}

\CommentTok{\# Run SMC estimator multiple times}
\NormalTok{smc.replicate }\OtherTok{\textless{}{-}} \ControlFlowTok{function}\NormalTok{(}\AttributeTok{reps=}\DecValTok{5000}\NormalTok{, }\AttributeTok{n=}\DecValTok{1000}\NormalTok{) \{}
\NormalTok{  estimates }\OtherTok{\textless{}{-}} \FunctionTok{sapply}\NormalTok{(}\DecValTok{1}\SpecialCharTok{:}\NormalTok{reps, }\ControlFlowTok{function}\NormalTok{(i) }\FunctionTok{smc.estimator}\NormalTok{(n))}
  \FunctionTok{return}\NormalTok{(estimates)}
\NormalTok{\}}

\CommentTok{\# Generate samples}
\NormalTok{is.results }\OtherTok{\textless{}{-}} \FunctionTok{is.replicate}\NormalTok{(}\AttributeTok{reps=}\DecValTok{1000}\NormalTok{)}
\NormalTok{smc.results }\OtherTok{\textless{}{-}} \FunctionTok{smc.replicate}\NormalTok{(}\AttributeTok{reps=}\DecValTok{1000}\NormalTok{)}

\CommentTok{\# Calculate the mean and variance}
\NormalTok{mean.is }\OtherTok{\textless{}{-}} \FunctionTok{mean}\NormalTok{(is.results)}
\NormalTok{mean.smc }\OtherTok{\textless{}{-}} \FunctionTok{mean}\NormalTok{(smc.results)}
\NormalTok{var.is }\OtherTok{\textless{}{-}} \FunctionTok{var}\NormalTok{(is.results)}
\NormalTok{var.smc }\OtherTok{\textless{}{-}} \FunctionTok{var}\NormalTok{(smc.results)}

\CommentTok{\# Print the results}
\FunctionTok{print}\NormalTok{(}\FunctionTok{paste}\NormalTok{(}\StringTok{"Average IS Estimate:"}\NormalTok{, }\FunctionTok{round}\NormalTok{(mean.is, }\DecValTok{5}\NormalTok{)))}
\end{Highlighting}
\end{Shaded}

\begin{verbatim}
## [1] "Average IS Estimate: 0.52653"
\end{verbatim}

\begin{Shaded}
\begin{Highlighting}[]
\FunctionTok{print}\NormalTok{(}\FunctionTok{paste}\NormalTok{(}\StringTok{"Average SMC Estimate:"}\NormalTok{, }\FunctionTok{round}\NormalTok{(mean.smc, }\DecValTok{5}\NormalTok{)))}
\end{Highlighting}
\end{Shaded}

\begin{verbatim}
## [1] "Average SMC Estimate: 0.527"
\end{verbatim}

\begin{Shaded}
\begin{Highlighting}[]
\FunctionTok{print}\NormalTok{(}\FunctionTok{paste}\NormalTok{(}\StringTok{"Variance of IS Estimates:"}\NormalTok{, }\FunctionTok{format}\NormalTok{(var.is, }\AttributeTok{scientific=}\ConstantTok{TRUE}\NormalTok{, }\AttributeTok{digits=}\DecValTok{5}\NormalTok{)))}
\end{Highlighting}
\end{Shaded}

\begin{verbatim}
## [1] "Variance of IS Estimates: 1.2518e-04"
\end{verbatim}

\begin{Shaded}
\begin{Highlighting}[]
\FunctionTok{print}\NormalTok{(}\FunctionTok{paste}\NormalTok{(}\StringTok{"Variance of SMC Estimates:"}\NormalTok{, }\FunctionTok{format}\NormalTok{(var.smc, }\AttributeTok{scientific=}\ConstantTok{TRUE}\NormalTok{, }\AttributeTok{digits=}\DecValTok{5}\NormalTok{)))}
\end{Highlighting}
\end{Shaded}

\begin{verbatim}
## [1] "Variance of SMC Estimates: 1.8229e-04"
\end{verbatim}

\begin{Shaded}
\begin{Highlighting}[]
\CommentTok{\# Plot the histograms}
\NormalTok{x.limits }\OtherTok{\textless{}{-}} \FunctionTok{range}\NormalTok{(}\FunctionTok{c}\NormalTok{(is.results, smc.results))}

\FunctionTok{par}\NormalTok{(}\AttributeTok{mfrow=}\FunctionTok{c}\NormalTok{(}\DecValTok{1}\NormalTok{,}\DecValTok{2}\NormalTok{))}

\FunctionTok{hist}\NormalTok{(smc.results, }\AttributeTok{breaks=}\DecValTok{40}\NormalTok{, }\AttributeTok{main=}\StringTok{"Simple Monte Carlo Estimates"}\NormalTok{,}
     \AttributeTok{xlab=}\StringTok{"Estimate Value"}\NormalTok{, }\AttributeTok{xlim=}\NormalTok{x.limits, }\AttributeTok{col=}\StringTok{"salmon"}\NormalTok{)}
\FunctionTok{abline}\NormalTok{(}\AttributeTok{v=}\NormalTok{mean.smc, }\AttributeTok{col=}\StringTok{"red"}\NormalTok{, }\AttributeTok{lwd=}\DecValTok{2}\NormalTok{)}

\FunctionTok{hist}\NormalTok{(is.results, }\AttributeTok{breaks=}\DecValTok{40}\NormalTok{, }\AttributeTok{main=}\StringTok{"Importance Sampling Estimates"}\NormalTok{,}
     \AttributeTok{xlab=}\StringTok{"Estimate Value"}\NormalTok{, }\AttributeTok{xlim=}\NormalTok{x.limits, }\AttributeTok{col=}\StringTok{"salmon"}\NormalTok{)}
\FunctionTok{abline}\NormalTok{(}\AttributeTok{v=}\NormalTok{mean.is, }\AttributeTok{col=}\StringTok{"blue"}\NormalTok{, }\AttributeTok{lwd=}\DecValTok{2}\NormalTok{)}
\end{Highlighting}
\end{Shaded}

\pandocbounded{\includegraphics[keepaspectratio]{Monte_Carlo_Methods_Sampling_Algorithms_files/figure-latex/unnamed-chunk-6-1.pdf}}

\begin{Shaded}
\begin{Highlighting}[]
\CommentTok{\# Set Parameters}
\NormalTok{N\_max }\OtherTok{\textless{}{-}} \DecValTok{1000}
\NormalTok{R }\OtherTok{\textless{}{-}} \DecValTok{100}       

\CommentTok{\# Create Matrices to Store Results}
\NormalTok{smc\_runs\_matrix }\OtherTok{\textless{}{-}} \FunctionTok{matrix}\NormalTok{(}\AttributeTok{nrow =}\NormalTok{ R, }\AttributeTok{ncol =}\NormalTok{ N\_max)}
\NormalTok{is\_runs\_matrix }\OtherTok{\textless{}{-}} \FunctionTok{matrix}\NormalTok{(}\AttributeTok{nrow =}\NormalTok{ R, }\AttributeTok{ncol =}\NormalTok{ N\_max)}

\CommentTok{\# Run Replications}
\ControlFlowTok{for}\NormalTok{ (i }\ControlFlowTok{in} \DecValTok{1}\SpecialCharTok{:}\NormalTok{R) \{}
  \CommentTok{\# SMC Run}
\NormalTok{  t.f }\OtherTok{\textless{}{-}} \FunctionTok{rt}\NormalTok{(N\_max, }\AttributeTok{df =} \DecValTok{2}\NormalTok{)}
\NormalTok{  phi\_vals\_smc }\OtherTok{\textless{}{-}} \FunctionTok{phi}\NormalTok{(t.f)}
\NormalTok{  smc\_runs\_matrix[i, ] }\OtherTok{\textless{}{-}} \FunctionTok{cumsum}\NormalTok{(phi\_vals\_smc) }\SpecialCharTok{/}\NormalTok{ (}\DecValTok{1}\SpecialCharTok{:}\NormalTok{N\_max)}
  
  \CommentTok{\# IS Run}
\NormalTok{  t.g }\OtherTok{\textless{}{-}} \FunctionTok{rnorm}\NormalTok{(N\_max)}
\NormalTok{  W }\OtherTok{\textless{}{-}} \FunctionTok{dt}\NormalTok{(t.g, }\AttributeTok{df =} \DecValTok{2}\NormalTok{) }\SpecialCharTok{/} \FunctionTok{dnorm}\NormalTok{(t.g)}
\NormalTok{  phi\_vals\_is }\OtherTok{\textless{}{-}} \FunctionTok{phi}\NormalTok{(t.g) }\SpecialCharTok{*}\NormalTok{ W}
\NormalTok{  is\_runs\_matrix[i, ] }\OtherTok{\textless{}{-}} \FunctionTok{cumsum}\NormalTok{(phi\_vals\_is) }\SpecialCharTok{/}\NormalTok{ (}\DecValTok{1}\SpecialCharTok{:}\NormalTok{N\_max)}
\NormalTok{\}}

\CommentTok{\# Calculate Plotting Bands}
\NormalTok{smc\_mean\_line }\OtherTok{\textless{}{-}}\NormalTok{ smc\_runs\_matrix[}\DecValTok{1}\NormalTok{, ]}
\NormalTok{smc\_min\_band }\OtherTok{\textless{}{-}} \FunctionTok{apply}\NormalTok{(smc\_runs\_matrix, }\DecValTok{2}\NormalTok{, min, }\AttributeTok{na.rm=}\ConstantTok{TRUE}\NormalTok{)}
\NormalTok{smc\_max\_band }\OtherTok{\textless{}{-}} \FunctionTok{apply}\NormalTok{(smc\_runs\_matrix, }\DecValTok{2}\NormalTok{, max, }\AttributeTok{na.rm=}\ConstantTok{TRUE}\NormalTok{)}

\NormalTok{is\_mean\_line }\OtherTok{\textless{}{-}}\NormalTok{ is\_runs\_matrix[}\DecValTok{1}\NormalTok{, ]}
\NormalTok{is\_min\_band }\OtherTok{\textless{}{-}} \FunctionTok{apply}\NormalTok{(is\_runs\_matrix, }\DecValTok{2}\NormalTok{, min, }\AttributeTok{na.rm=}\ConstantTok{TRUE}\NormalTok{)}
\NormalTok{is\_max\_band }\OtherTok{\textless{}{-}} \FunctionTok{apply}\NormalTok{(is\_runs\_matrix, }\DecValTok{2}\NormalTok{, max, }\AttributeTok{na.rm=}\ConstantTok{TRUE}\NormalTok{)}

\NormalTok{iterations }\OtherTok{\textless{}{-}} \DecValTok{1}\SpecialCharTok{:}\NormalTok{N\_max}

\CommentTok{\# Plotting}
\FunctionTok{par}\NormalTok{(}\AttributeTok{mfrow =} \FunctionTok{c}\NormalTok{(}\DecValTok{1}\NormalTok{, }\DecValTok{2}\NormalTok{))}

\CommentTok{\# Plot 1: SMC}
\NormalTok{smc\_ylim }\OtherTok{\textless{}{-}} \FunctionTok{range}\NormalTok{(}\FunctionTok{c}\NormalTok{(smc\_min\_band, smc\_max\_band), }\AttributeTok{na.rm =} \ConstantTok{TRUE}\NormalTok{, }\AttributeTok{finite =} \ConstantTok{TRUE}\NormalTok{)}
\FunctionTok{plot}\NormalTok{(iterations, smc\_mean\_line, }\AttributeTok{type =} \StringTok{"n"}\NormalTok{, }\AttributeTok{ylim =}\NormalTok{ smc\_ylim,}
     \AttributeTok{main =} \StringTok{"SMC (Sampling from t\_2)"}\NormalTok{, }\AttributeTok{xlab =} \StringTok{"Iteration (n)"}\NormalTok{, }\AttributeTok{ylab =} \StringTok{"Estimate"}\NormalTok{)}
\FunctionTok{polygon}\NormalTok{(}\FunctionTok{c}\NormalTok{(iterations, }\FunctionTok{rev}\NormalTok{(iterations)), }\FunctionTok{c}\NormalTok{(smc\_min\_band, }\FunctionTok{rev}\NormalTok{(smc\_max\_band)),}
        \AttributeTok{col =} \StringTok{"grey80"}\NormalTok{, }\AttributeTok{border =} \ConstantTok{NA}\NormalTok{)}
\FunctionTok{lines}\NormalTok{(iterations, smc\_mean\_line, }\AttributeTok{col =} \StringTok{"black"}\NormalTok{, }\AttributeTok{lwd =} \DecValTok{1}\NormalTok{)}

\CommentTok{\# Plot 2: IS}
\NormalTok{is\_ylim\_clipped }\OtherTok{\textless{}{-}} \FunctionTok{range}\NormalTok{(}\FunctionTok{c}\NormalTok{(}\FunctionTok{quantile}\NormalTok{(is\_min\_band, }\FloatTok{0.05}\NormalTok{, }\AttributeTok{na.rm =} \ConstantTok{TRUE}\NormalTok{), }
                           \FunctionTok{quantile}\NormalTok{(is\_max\_band, }\FloatTok{0.95}\NormalTok{, }\AttributeTok{na.rm =} \ConstantTok{TRUE}\NormalTok{),}
\NormalTok{                           smc\_ylim), }\AttributeTok{na.rm=}\ConstantTok{TRUE}\NormalTok{, }\AttributeTok{finite=}\ConstantTok{TRUE}\NormalTok{)}
\FunctionTok{plot}\NormalTok{(iterations, is\_mean\_line, }\AttributeTok{type =} \StringTok{"n"}\NormalTok{, }\AttributeTok{ylim =}\NormalTok{ is\_ylim\_clipped,}
     \AttributeTok{main =} \StringTok{"IS (using N(0,1) proposal)"}\NormalTok{, }\AttributeTok{xlab =} \StringTok{"Iteration (n)"}\NormalTok{, }\AttributeTok{ylab =} \StringTok{"Estimate"}\NormalTok{)}
\FunctionTok{polygon}\NormalTok{(}\FunctionTok{c}\NormalTok{(iterations, }\FunctionTok{rev}\NormalTok{(iterations)), }\FunctionTok{c}\NormalTok{(is\_min\_band, }\FunctionTok{rev}\NormalTok{(is\_max\_band)),}
        \AttributeTok{col =} \StringTok{"grey80"}\NormalTok{, }\AttributeTok{border =} \ConstantTok{NA}\NormalTok{)}
\FunctionTok{lines}\NormalTok{(iterations, is\_mean\_line, }\AttributeTok{col =} \StringTok{"black"}\NormalTok{, }\AttributeTok{lwd =} \DecValTok{1}\NormalTok{)}
\end{Highlighting}
\end{Shaded}

\pandocbounded{\includegraphics[keepaspectratio]{Monte_Carlo_Methods_Sampling_Algorithms_files/figure-latex/unnamed-chunk-7-1.pdf}}

\end{document}
